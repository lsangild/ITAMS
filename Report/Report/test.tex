% !TEX root = ../main.tex
% !TEX spellcheck = en_GB

\chapter{Test}
In this section an short introduction to how this project has used test from beginning to the end will be described.

\section{Module test}
The goal of the module tests is to determine if the used modules are able to perform the actions demanded by the \nameref{chap:requirements}.

Module tests has been performed on the \GPS GPS module, the \SARA GSM module and the \SDsock µSD-card socket with µSD-card inserted.

\subsection{GPS \GPS}
The GPS module is tested by connecting it to a USB to UART module and sending the appropriate UBX commands. The setting is outdoors as to achieve the most precise localization.

Before this, the command CFG-PRT (Port Configuration for UART) \cite[p.~119-120]{NEO7_proto}, must be sent to the GPS module.

\begin{table}[H]
	\centering
	\begin{tabularx}{\textwidth}{p{4.3cm} X X}
		\toprule
		\textbf{Action} & \textbf{Expected outcome} & \textbf{Outcome} \\
		\midrule
		Send \textit{0xB5620107000008190D0A} (UBX-NAV-PVT/Get location) from computer to GPS module via USB to UART with \num{9600} baud. & Receive \num{92} bytes according to \cite[p.~160-161]{NEO7_proto}. & \\
		\midrule
		Input coordinates achieved from above test into Google Maps. & Pin is placed at the location above command was issued at. & \\
		\bottomrule
	\end{tabularx}
	\caption{Module test of \GPS GPS module.}
	\label{AT:modGPS}
\end{table}

\subsection{GSM \SARA}
The module test of the GSM \SARA tests if the module is able to send a simple string to a known server.

The used test command is
\begin{quote}
	AT+USOST=\textit{SOCK},"\textit{IP}",\textit{PORT},11@Hello World
\end{quote}
Where \textit{SOCK} is decided by the \SARA module, and \textit{IP} and \textit{PORT} are decided by the server.

\begin{table}[H]
	\centering
	\begin{tabularx}{\textwidth}{p{4.3cm} X X}
		\toprule
		\textbf{Action} & \textbf{Expected outcome} & \textbf{Outcome} \\
		\midrule
		Send test command, via a USB to UART module, to \MKR coded with an Arduino Sketch forwarding the input to the \SARA. & The server log shows "Hello World". & \\
		\bottomrule
	\end{tabularx}
	\caption{Module test of \SARA GSM module.}
	\label{AT:modGSM}
\end{table}

\subsection{\SDsock socket and µSD-card}
Before the test, use \textit{dd} on a Linux device, to clear the µSD-card with \mintinline{bash}{dd if=/dev/zero of=/dev/sdX}, where sdX is the µSD-card.

\begin{table}[H]
	\centering
	\begin{tabularx}{\textwidth}{p{4.3cm} X X}
		\toprule
		\textbf{Action} & \textbf{Expected outcome} & \textbf{Outcome} \\
		\midrule
		Use an Arduino Sketch driver for the \MKR to save the string "Hello World" to the µSD-card via SPI.
		Remove the µSD-card from the \SDsock socket and read it on a Linux device using \textit{dd} to a file. & The output file shows "Hello World". & \\
		\bottomrule
	\end{tabularx}
	\caption{Module test of \SDsock socket and µSD-card.}
	\label{AT:modSD}
\end{table}

\section{Integration test}
The integration tests are much like the module tests, except for the \MKR being the facilitator of the data.

\subsection{\MKR with \GPS}
Before this, the command CFG-PRT (Port Configuration for UART) \cite[p.~119-120]{NEO7_proto}, must be sent to the GPS module.

\begin{table}[H]
	\centering
	\begin{tabularx}{\textwidth}{p{4.3cm} X X}
		\toprule
		\textbf{Action} & \textbf{Expected outcome} & \textbf{Outcome} \\
		\midrule
		Send \textit{0xB5620107000008190D0A} (UBX-NAV-PVT/Get location) from \MKR, with debugger on, to GPS module via UART with \num{9600} baud. & Receive \num{92} bytes according to \cite[p.~160-161]{NEO7_proto}. & \\
		\midrule
		Input coordinates achieved from above test into Google Maps. & Pin is placed at the location above command was issued at. & \\
		\bottomrule
	\end{tabularx}
	\caption{Integration test of \MKR and \GPS GPS module.}
	\label{AT:intGPS}
\end{table}

\subsection{\MKR with \SARA}
The used test command is
\begin{quote}
	AT+USOST=\textit{SOCK},"\textit{IP}",\textit{PORT},11@Hello World
\end{quote}
Where \textit{SOCK} is decided by the \SARA module, and \textit{IP} and \textit{PORT} are decided by the server.

\begin{table}[H]
	\centering
	\begin{tabularx}{\textwidth}{p{4.3cm} X X}
		\toprule
		\textbf{Action} & \textbf{Expected outcome} & \textbf{Outcome} \\
		\midrule
		Send test command from \MKR to \SARA via UART. & The server log shows "Hello World". & \\
		\bottomrule
	\end{tabularx}
	\caption{Integration test of \MKR and \SARA GSM module.}
	\label{AT:intGSM}
\end{table}

\subsection{\MKR with \SDsock and µSD-card}
Before the test, use \textit{dd} on a Linux device, to clear the µSD-card with \mintinline{bash}{dd if=/dev/zero of=/dev/sdX}, where sdX is the µSD-card.

\begin{table}[H]
	\centering
	\begin{tabularx}{\textwidth}{p{4.3cm} X X}
		\toprule
		\textbf{Action} & \textbf{Expected outcome} & \textbf{Outcome} \\
		\midrule
		Use the \MKR to save the string "Hello World" to the µSD-card via SPI.
		Remove the µSD-card from the \SDsock socket and read it on a Linux device using \textit{dd} to a file. & The output file shows "Hello World". & \\
		\bottomrule
	\end{tabularx}
	\caption{Integration test of \MKR and \SDsock socket and µSD-card.}
	\label{AT:intSD}
\end{table}

\section{Acceptance Test}
The goal of the acceptance test is to test the system as a whole.
Due to time constraints, two acceptance tests are produced, one with and one without the \SDsock and µSD-card.

Acceptance tests are carried out outdoors to ensure good GPS signal quality.

\begin{table}[H]
	\centering
	\begin{tabularx}{\textwidth}{p{4.3cm} X X}
		\toprule
		\textbf{Action} & \textbf{Expected outcome} & \textbf{Outcome} \\
		\midrule
		Use the \MKR to save the string "Hello World" to the µSD-card via SPI.
		Remove the µSD-card from the \SDsock socket and read it on a Linux device using \textit{dd} to a file. & The output file shows "Hello World". & \\
		\bottomrule
	\end{tabularx}
	\caption{Acceptance test of \systemName with  \SDsock socket and µSD-card.}
	\label{AT:withSD}
\end{table}

\FloatBarrier