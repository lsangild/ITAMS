% !TEX root = ../main.tex
% !TEX spellcheck = en_GB

\chapter{Conclusion}
\label{chap:conclusion}
The goal of the project has been to create a \systemName able to locate itself and send the location to a remote server, possibly sending at specified intervals to conserve power.
The final prototype incorporates an ARM µC, a GSM cellular module and a GPS module.

A great deal of work went into getting to know the \SAMD µC, this was a hindrance to the work flow.
As the \SAMD is built on the SAM structure and not the AVR structure, a lot of new registers had to be understood.
A solution to this could have been to use Arduino Sketches or the Atmel ASF framework, but the team felt this might have led to less learning from the project.
Another solution to the unknown architecture could have been to use the known ATmega2560.

The interfacing with the GSM and GPS modules were fairly straightforward, but was again hindered by the above described lack of knowlede of the \SAMD µC.
This showed in an unreliable UART communication, which was not made stable until the very end of the project.

An issue encountered with the GSM module was the need for sudden bursts of current, the antenna needed.
These bursts are up to \SI{2}{\ampere}, which a standard USB power supply is not able to supply.
The solution was to attach a \SI{3.8}{\volt} mobile phone battery to the \MKR.

The final prototype is able to locate itself.
It is not able to save location data over extended periods without internet access.
The GPS data gathered is sent to a remote server, available for backup or analysis.

\FloatBarrier