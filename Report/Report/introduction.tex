% !TEX root = ../main.tex
% !TEX spellcheck = en_GB

\chapter{Introduction}
\label{sec:introduction}
I den moderne verden er der et stort behov for sporing af forskellig karakter. Der er stort potentiale for sporing gennem gps, af biler, containere eller sågar demente ældre der kan finde på at udvandre fra plejehjem. 
Gruppen har derfor besluttet at udfordrer sig selv ved at implementere et batteri drevet system der lagre locations data løbende, og ved intervaler sender denne data tilbage til en central server som permanent lagre dataen. Systemet kan potientelt bruges til en stor række applikationer, men dette projekt forkusere blot på at få det mest baselle op at kører.
Der er blevet lånt en \MKR, som skal fungere som hoved enhed for systemet, dette er både da den besider et allerede indbygget GSM modem, samt også fordi den som prototype skal testes af skolen.
\fxnote{Translate to English}

\section{Project Description}
\label{sec:projectDescription}


\section{Project Delimitations}
\label{sec:delimitations}
The \nameref{sec:projectDescription} is the groups vision of the ideal solution to the problem outlined in the \nameref{sec:introduction}.  
To give the best possible view of the groups capabilities in developing such a solution, the most core parts of the project will have the highest priority, hence some parts will be excluded from this version of the project.

\section{Terminology}
\label{sec:terminology}
Below in \cref{tab:terminology} is shown a list of terms used throughout the report describing each name for clarification purposes.

\begin{table}[H]
	\centering
	\begin{tabularx}{0.8\textwidth}{l X}
		\toprule
		\textbf{Name} & \textbf{Description} \\
		\midrule
		&\\
		\bottomrule
	\end{tabularx}
	\caption{List of terminologies.}
	\label{tab:terminology}
\end{table}